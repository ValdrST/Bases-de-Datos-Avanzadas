\documentclass[journal]{IEEEtran}
\usepackage[english]{babel}

\usepackage{amssymb, amsmath} %Paquetes matemáticos de la American Mathematical 
\usepackage[utf8]{inputenc}
\usepackage{graphicx}
\usepackage{float}
\usepackage{hyperref}
\usepackage{listings}
\usepackage{xcolor}

\definecolor{codegreen}{rgb}{0,0.6,0}
\definecolor{codegray}{rgb}{0.5,0.5,0.5}
\definecolor{codepurple}{rgb}{0.58,0,0.82}
\definecolor{backcolour}{rgb}{0.95,0.95,0.92}
% Definicio de estilo para el codigo fuente que se cita
\lstdefinestyle{mystyle}{
    backgroundcolor=\color{backcolour},   
    commentstyle=\color{codegreen},
    keywordstyle=\color{magenta},
    numberstyle=\tiny\color{codegray},
    stringstyle=\color{codepurple},
    basicstyle=\ttfamily\footnotesize,
    breakatwhitespace=false,         
    breaklines=true,                 
    captionpos=b,                    
    keepspaces=true,
    numbers=left,                    
    numbersep=5pt,                  
    showspaces=false,                
    showstringspaces=false,
    showtabs=false,                  
    tabsize=2,
}
\lstset{style=mystyle}

\renewcommand{\lstlistingname}{Código}

\ifCLASSINFOpdf

\else

\fi
\begin{document}

\title{Ejercicio 5 - tema 6 \\ Administración básica de tablespaces}
%
\author{Vicente Romero Andrade}

\markboth{Ejercicio 5 - tema 6 Administración básica de tablespaces, Julio~2021}%
{Shell \MakeLowercase{\textit{et al.}}: }
% The only time the second header will appear is for the odd numbered pages

\maketitle


\IEEEpeerreviewmaketitle

\section{Objetivo}
% The very first letter is a 2 line initial drop letter followed

\IEEEPARstart{E}{l} objetivo es comprender y familiarizarse con las diferentes opciones y configuraciones 
empleadas para crear tablespaces, así como la asignación de tablespaces por default a un usuario en particular.
\section{Desarrollo}
\subsection{sentencias}
\begin{lstlisting}[language=sql, caption=s-00-crea-tablespace.sql,label={lst:codigo1}]
  --@Autor: Vicente Romero Andrade
  whenever sqlerror exit rollback
  set serveroutput on
  connect sys/system2 as sysdba
  --A
  create tablespace store_tbs1 
    datafile '/u01/app/oracle/oradata/VRABDA2/store_tbs01.dbf' 
    size 20M
    autoextent off
    extent management local autoallocate
    segment space management auto;
  --B
  create tablespace store_tbs_multiple
    datafile '/u01/app/oracle/oradata/VRABDA2/store_tbs_multiple_01.dbf' size 10M,
    '/u01/app/oracle/oradata/VRABDA2/store_tbs_multiple_02.dbf' size 10M,
    '/u01/app/oracle/oradata/VRABDA2/store_tbs_multiple_03.dbf' size 10M;
  --C
  create tablespace store_tbs_custom
    datafile '/u01/app/oracle/oradata/VRABDA2/store_tbs_custom_01.dbf' size 10M reuse
    autoextend on next 5m maxsize 30m
    blocksize 8k
    nologging
    offline
    extent management local uniform size 64k
    segment space management auto;
  --D
  CREATE USER store_user IDENTIFIED BY store_user quota unlimited on store_tbs1 default tablespace store_tbs1;
  GRANT CONNECT TO store_user;
  GRANT CREATE TABLE TO store_user;
  
  whenever sqlerror continue
  
\end{lstlisting}
\section{Conclusiones}
Se reviso la sintaxis basica para crear tablespaces con configuraciones particulares, asi como la asignación del usuario 
que hara uso de las mismas.
\ifCLASSOPTIONcaptionsoff
  \newpage

\fi

\end{document}
