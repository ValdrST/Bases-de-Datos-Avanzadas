\documentclass[journal]{IEEEtran}
\usepackage[english]{babel}

\usepackage{amssymb, amsmath} %Paquetes matemáticos de la American Mathematical 
\usepackage[utf8]{inputenc}
\usepackage{graphicx}
\usepackage{float}
\usepackage{hyperref}
\usepackage{listings}
\usepackage{xcolor}

\definecolor{codegreen}{rgb}{0,0.6,0}
\definecolor{codegray}{rgb}{0.5,0.5,0.5}
\definecolor{codepurple}{rgb}{0.58,0,0.82}
\definecolor{backcolour}{rgb}{0.95,0.95,0.92}
% Definicio de estilo para el codigo fuente que se cita
\lstdefinestyle{mystyle}{
    backgroundcolor=\color{backcolour},   
    commentstyle=\color{codegreen},
    keywordstyle=\color{magenta},
    numberstyle=\tiny\color{codegray},
    stringstyle=\color{codepurple},
    basicstyle=\ttfamily\footnotesize,
    breakatwhitespace=false,         
    breaklines=true,                 
    captionpos=b,                    
    keepspaces=true,
    numbers=left,                    
    numbersep=5pt,                  
    showspaces=false,                
    showstringspaces=false,
    showtabs=false,                  
    tabsize=2,
}
\lstset{style=mystyle}

\renewcommand{\lstlistingname}{Código}

\ifCLASSINFOpdf

\else

\fi
\begin{document}

\title{Ejercicio 2 - tema 7 \\ Almacenamiento en data files}
%
\author{Vicente Romero Andrade}

\markboth{Ejercicio 2 - tema 7 Almacenamiento en data files, Julio~2021}%
{Shell \MakeLowercase{\textit{et al.}}: }
% The only time the second header will appear is for the odd numbered pages

\maketitle


\IEEEpeerreviewmaketitle

\section{Objetivo}
% The very first letter is a 2 line initial drop letter followed

\IEEEPARstart{E}{l} objetivo es poner en práctica las tareas de administración 
que permitan el almacenamiento de datos de una tabla en un tablespace y data file 
específico configurado previamente

\section{Desarrollo}
\subsection{sentencias}
\begin{lstlisting}[language=sql, caption=s-00-datafile.sql,label={lst:codigo1}]
  whenever sqlerror exit rollback
  set serveroutput on
  connect sys/system2 as sysdba
  --A
  SELECT
    FILE_NAME,
    FILE_ID,
    TRUNC((BYTES/(1024*1024)),2) SIZE_MB
  FROM DBA_DATA_FILES WHERE TABLESPACE_NAME ='STORE_TBS_MULTIPLE';
  --B
  SELECT
         TRUNC((SUM(DF.BYTES)-NVL(SUM(S.BYTES),0))/(1024*1024),2) MB_LIBRES,
         (SUM(DF.BLOCKS)-NVL(SUM(S.BLOCKS),0)) BLOQUES_DISPONIBLES
  FROM DBA_DATA_FILES DF
      LEFT JOIN DBA_SEGMENTS S
          ON S.TABLESPACE_NAME = DF.TABLESPACE_NAME
  WHERE DF.TABLESPACE_NAME='STORE_TBS_MULTIPLE';
  --C
  --CREATE USER VRA_TBS_MULTIPLE IDENTIFIED BY VRA_TBS_MULTIPLE 
    --quota unlimited on store_tbs_multiple 
    --default tablespace store_tbs_multiple;
  --D
  declare
    v_count number;
    v_username varchar2(30) := 'VRA_TBS_MULTIPLE';
    v_table varchar2(30) := 'VRA_TBS_MULTIPLE';
  begin
    --Verificar si la table existe
    select count(*) into v_count
    from all_tables
    where table_name = v_table
    and owner = v_username;
  
    if v_count > 0 then
      execute immediate 'drop table '||v_username||'.'||v_table;
    end if;
    execute immediate 'create table '||v_username||'.'||v_table||' (
      str char(1024 byte)
    ) segment creation immediate';
  end;
  /
  --E
  SELECT DF.FILE_NAME,
         DF.FILE_ID,
         COUNT(DE.SEGMENT_NAME) NUMERO_EXTENSIONES,
         SUM(DE.BYTES/(1024*1024)) TOTAL_MB,
         SUM(DE.BLOCKS) BLOQUES_RESERVADOS
      FROM DBA_SEGMENTS DS
  JOIN DBA_DATA_FILES DF
      ON DS.HEADER_FILE = DF.FILE_ID
  JOIN DBA_DATA_FILES DF
      ON DS.HEADER_FILE = DF.FILE_ID
  JOIN DBA_EXTENTS DE
      ON DS.SEGMENT_NAME = DE.SEGMENT_NAME
  WHERE DS.SEGMENT_NAME like '%VRA_TBS_MULTIPLE%'
  GROUP BY DF.FILE_NAME, DF.FILE_ID;
  --F
  declare
    v_count number := 0;
  begin
    while v_count < 512 loop
    insert into VRA_TBS_MULTIPLE.VRA_TBS_MULTIPLE(str) values('$');
    v_count := v_count + 1;
    end loop;
    commit;
  end;
  /
  --G
  SELECT DF.FILE_NAME,
         DF.FILE_ID,
         COUNT(DE.SEGMENT_NAME) NUMERO_EXTENSIONES,
         SUM(DE.BYTES/(1024*1024)) TOTAL_MB,
         SUM(DE.BLOCKS) BLOQUES_RESERVADOS
      FROM DBA_SEGMENTS DS
  JOIN DBA_DATA_FILES DF
      ON DS.HEADER_FILE = DF.FILE_ID
  JOIN DBA_DATA_FILES DF
      ON DS.HEADER_FILE = DF.FILE_ID
  JOIN DBA_EXTENTS DE
      ON DS.SEGMENT_NAME = DE.SEGMENT_NAME
  WHERE DS.SEGMENT_NAME like '%VRA_TBS_MULTIPLE%'
  GROUP BY DF.FILE_NAME, DF.FILE_ID;
  --H
  declare
    v_count number := 0;
  begin
    while v_count < 512*5 loop
    insert into VRA_TBS_MULTIPLE.VRA_TBS_MULTIPLE(str) values('$');
    v_count := v_count + 1;
    end loop;
    commit;
  end;
  /
  SELECT DF.FILE_NAME,
         DF.FILE_ID,
         COUNT(DE.SEGMENT_NAME) NUMERO_EXTENSIONES,
         SUM(DE.BYTES/(1024*1024)) TOTAL_MB,
         SUM(DE.BLOCKS) BLOQUES_RESERVADOS
      FROM DBA_SEGMENTS DS
  JOIN DBA_DATA_FILES DF
      ON DS.HEADER_FILE = DF.FILE_ID
  JOIN DBA_DATA_FILES DF
      ON DS.HEADER_FILE = DF.FILE_ID
  JOIN DBA_EXTENTS DE
      ON DS.SEGMENT_NAME = DE.SEGMENT_NAME
  WHERE DS.SEGMENT_NAME like '%VRA_TBS_MULTIPLE%'
  GROUP BY DF.FILE_NAME, DF.FILE_ID;
  --I
  SELECT
         TRUNC((SUM(DF.BYTES)-NVL(SUM(S.BYTES),0))/(1024*1024),2) MB_LIBRES,
         (SUM(DF.BLOCKS)-NVL(SUM(S.BLOCKS),0)) BLOQUES_DISPONIBLES
  FROM DBA_DATA_FILES DF
      LEFT JOIN DBA_SEGMENTS S
          ON S.TABLESPACE_NAME = DF.TABLESPACE_NAME
  WHERE DF.TABLESPACE_NAME='STORE_TBS_MULTIPLE';
  
  whenever sqlerror continue
\end{lstlisting}

\begin{figure}[H]
  \centering
  \includegraphics[scale=.25]{captura_a.png}
   \caption{Salida punto A}
   \label{fig:validador_1}
\end{figure}
\begin{figure}[H]
  \centering
  \includegraphics[scale=.50]{captura_b.png}
   \caption{Salida punto B}
   \label{fig:validador_2}
\end{figure}
\begin{figure}[H]
  \centering
  \includegraphics[scale=.50]{captura_1.png}
   \caption{Salida punto E}
   \label{fig:validador_2}
\end{figure}

\section{Conclusiones}
Se encontro una forma eficiente de consultar los segmentos creados en una tabla, 
el unico incoveniente es que estos dependen de que sean creados con el nombre de la tabla 
para su busqueda.
\ifCLASSOPTIONcaptionsoff
  \newpage

\fi

\end{document}
