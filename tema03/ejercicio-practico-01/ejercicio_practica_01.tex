\documentclass[journal]{IEEEtran}
\usepackage[english]{babel}

\usepackage{amssymb, amsmath} %Paquetes matemáticos de la American Mathematical 
\usepackage[utf8]{inputenc}
\usepackage{graphicx}
\usepackage{float}
\usepackage{hyperref}
\usepackage{listings}
\usepackage{xcolor}

\definecolor{codegreen}{rgb}{0,0.6,0}
\definecolor{codegray}{rgb}{0.5,0.5,0.5}
\definecolor{codepurple}{rgb}{0.58,0,0.82}
\definecolor{backcolour}{rgb}{0.95,0.95,0.92}
% Definicio de estilo para el codigo fuente que se cita
\lstdefinestyle{mystyle}{
    backgroundcolor=\color{backcolour},   
    commentstyle=\color{codegreen},
    keywordstyle=\color{magenta},
    numberstyle=\tiny\color{codegray},
    stringstyle=\color{codepurple},
    basicstyle=\ttfamily\footnotesize,
    breakatwhitespace=false,         
    breaklines=true,                 
    captionpos=b,                    
    keepspaces=true,
    numbers=left,                    
    numbersep=5pt,                  
    showspaces=false,                
    showstringspaces=false,
    showtabs=false,                  
    tabsize=2,
}
\lstset{style=mystyle}

\renewcommand{\lstlistingname}{Código}

\ifCLASSINFOpdf

\else

\fi

\hyphenation{op-tical net-works semi-conduc-tor}


\begin{document}

\title{Ejercicio 1 - tema 3 \\ Procesos startup y shutdown de una base de datos.}
%
\author{Vicente Romero Andrade}

\markboth{Ejercicio 1 - tema 3 Procesos startup y shutdown de una base de datos., Junio~2021}%
{Shell \MakeLowercase{\textit{et al.}}: }
% The only time the second header will appear is for the odd numbered pages

\maketitle


\IEEEpeerreviewmaketitle

\section{Objetivo}
% The very first letter is a 2 line initial drop letter followed

\IEEEPARstart{E}{l} objetivo es, Comprender las características, acciones realizadas y 
comportamiento de las diferentes etapas y modos que tiene una base de datos tanto para iniciar como para detener una instancia

\section{Desarrollo}
\subsection{C1.  Incluir la tabla de respuestas, agregar una breve explicación y/o justificación}
\begin{table}[H]
  \centering
  \begin{tabular}{||c |c| c||} 
   \hline
   Pregunta & Respuesta(inciso) &Explicación/justificación.\\ [1.0ex] 
   \hline
   P01 &  &  \\ 
   P02 &  &  \\ 
   P03 &  &  \\ 
   P04 &  &  \\ 
   P05 &  &  \\ 
   P06 &  &  \\ 
   P07 &  &  \\ 
   P08 &  &  \\ 
   P09 &  &  \\ 
   P10 &  &  \\ 
   P11 &  &  \\ 
   P12 &  &  \\ 
   P13 &  &  \\  [1ex] 
   \hline
  \end{tabular}
  \caption{Tabla De respuestas}
  \label{tabla:1}
  \end{table}

\section{Conclusiones}
En este ejercicio se vieron los componentes de la base de datos que 
son necesarios conocer antes de poder crear una nueva base, estos
son importantes conocerlos para poder hacer estimaciones de los 
recursos que van a ser necesarios más adelante. La única dificultad
fue hacer que coincidiera el timezone\_offset con el del validador,
podría añadirse una sección para cambiar la zona horaria de la base de datos.
\ifCLASSOPTIONcaptionsoff
  \newpage

\fi

\end{document}
