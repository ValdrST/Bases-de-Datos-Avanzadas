\documentclass[journal]{IEEEtran}
\usepackage[english]{babel}

\usepackage{amssymb, amsmath} %Paquetes matemáticos de la American Mathematical 
\usepackage[utf8]{inputenc}
\usepackage{graphicx}
\usepackage{float}
\usepackage{hyperref}
\usepackage{listings}
\usepackage{xcolor}

\definecolor{codegreen}{rgb}{0,0.6,0}
\definecolor{codegray}{rgb}{0.5,0.5,0.5}
\definecolor{codepurple}{rgb}{0.58,0,0.82}
\definecolor{backcolour}{rgb}{0.95,0.95,0.92}
% Definicio de estilo para el codigo fuente que se cita
\lstdefinestyle{mystyle}{
    backgroundcolor=\color{backcolour},   
    commentstyle=\color{codegreen},
    keywordstyle=\color{magenta},
    numberstyle=\tiny\color{codegray},
    stringstyle=\color{codepurple},
    basicstyle=\ttfamily\footnotesize,
    breakatwhitespace=false,         
    breaklines=true,                 
    captionpos=b,                    
    keepspaces=true,
    numbers=left,                    
    numbersep=5pt,                  
    showspaces=false,                
    showstringspaces=false,
    showtabs=false,                  
    tabsize=2,
}
\lstset{style=mystyle}

\renewcommand{\lstlistingname}{Código}

\ifCLASSINFOpdf

\else

\fi
\begin{document}

\title{Proyecto final \\ Bases de Datos Avanzadas}
%
\author{Vicente Romero Andrade}

\markboth{Proyecto final Bases de Datos Avanzadas, Agosto~2021}%
{Shell \MakeLowercase{\textit{et al.}}: }
% The only time the second header will appear is for the odd numbered pages

\maketitle


\IEEEpeerreviewmaketitle

\section{Objetivo}
% The very first letter is a 2 line initial drop letter followed

\IEEEPARstart{E}{l} objetivo de este proyecto es poner en práctica lo visto a lo largo 
del curso.

\section{Creación de la base de datos}
\subsection{Resumen de scripts creados}
\begin{table}[H]
  \centering
  \resizebox{\linewidth}{!}{\begin{tabular}{|c | c | c|} 
   \hline
   Num. Script & Nombre del script & Descripción \\ [0.5ex] 
   \hline
   1 & s-01-crea-directorios.sh & Crea los directorios que simularan los discos asdasdasdasdasda\\ 
   \hline
   2 & s-02-crea-pfile.sh & Crea el pfile y el archivo passwords \\
   \hline
   3 & s-03-crear-spfile.sql & Crea el archivo de spfile  \\
   \hline
   4 & s-04-crear-base.sql & Crea la base de datos asi como sus configuraciones basicas \\
   \hline
   5 & s-05-crear-diccionario-datos.sql & Crea el diccionario de datos de la base \\
   \hline
   6 & s-06-crear-tablespaces.sql & Crea los tablespaces de la base de datos \\ [1ex] 
   \hline
  \end{tabular}}
  \caption{Resumen scripts creados}
  \label{tabla:1}
\end{table}
\subsection{Configuraciones de la base de datos}
\begin{table}[H]
  \centering
  \resizebox{\linewidth}{!}{\begin{tabular}{|c | c | c|} 
   \hline
   Configuración & Descripción \\ [0.5ex] 
   \hline
   Número y ubicación de los archivos de control & .\\ 
   \hline
   Propuesta de grupos de REDO & .\\ 
   \hline
   Propuesta de juego de caracteres & .\\ 
   \hline
   Tamaño del bloque de datos & .\\ 
   \hline
   Lista de parámetros que serán configurados al crear la base de datos &
    .\\
    \hline
    Archivo passwords & .\\ [1ex] 
   \hline
  \end{tabular}}
  \caption{Configuraciones de la base de datos}
  \label{tabla:2}
\end{table}
\subsection{Módulos del sistema}
\begin{table}[H]
  \centering
  \resizebox{\linewidth}{!}{\begin{tabular}{|c | c | c|} 
   \hline
   Nombre del módulo & Descripción & Usuario \\ [0.5ex] 
   \hline
   Administración de usuarios & Es el modulo que gestionara los datos de los usuarios asi como los pagos y planes de suscripción & admin\_usuario \\ 
   \hline
   Administración de multimedia & Es el modulo que gestionara los datos multimedia asi como el contenido relacionado a sus autores & admin\_multimedia \\  [1ex] 
   \hline
  \end{tabular}}
  \caption{Módulos del sistema}
  \label{tabla:3}
\end{table}
\subsection{Diseño lógico de la base de datos}
\begin{table}[H]
  \centering
  \resizebox{\linewidth}{!}{\begin{tabular}{|c | c | c|} 
   \hline
   Nombre del módulo & Descripción & Usuario \\ [0.5ex] 
   \hline
   Administración de usuarios & Es el modulo que gestionara los datos de los usuarios asi como los pagos y planes de suscripción & admin\_usuario \\ 
   \hline
   Administración de multimedia & Es el modulo que gestionara los datos multimedia asi como el contenido relacionado a sus autores & admin\_multimedia \\  [1ex] 
   \hline
  \end{tabular}}
  \caption{Módulos del sistema}
  \label{tabla:4}
\end{table}
\ifCLASSOPTIONcaptionsoff
  \newpage

\fi

\end{document}
