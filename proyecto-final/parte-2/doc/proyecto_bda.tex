\documentclass[journal]{IEEEtran}
\usepackage[english]{babel}

\usepackage{amssymb, amsmath} %Paquetes matemáticos de la American Mathematical 
\usepackage[utf8]{inputenc}
\usepackage{graphicx}
\usepackage{float}
\usepackage{hyperref}
\usepackage{listings}
\usepackage[table, svgnames, dvipsnames]{xcolor}
\usepackage{multirow}
\usepackage{supertabular}
\usepackage{longtable}
\newcolumntype{L}[1]{>{\raggedright\let\newline\\\arraybackslash\hspace{0pt}}m{#1}}
\definecolor{codegreen}{rgb}{0,0.6,0}
\definecolor{codegray}{rgb}{0.5,0.5,0.5}
\definecolor{codepurple}{rgb}{0.58,0,0.82}
\definecolor{backcolour}{rgb}{0.95,0.95,0.92}
% Definicio de estilo para el codigo fuente que se cita
\lstdefinestyle{mystyle}{
    backgroundcolor=\color{backcolour},   
    commentstyle=\color{codegreen},
    keywordstyle=\color{magenta},
    numberstyle=\tiny\color{codegray},
    stringstyle=\color{codepurple},
    basicstyle=\ttfamily\footnotesize,
    breakatwhitespace=false,         
    breaklines=true,                 
    captionpos=b,                    
    keepspaces=true,
    numbers=left,                    
    numbersep=5pt,                  
    showspaces=false,                
    showstringspaces=false,
    showtabs=false,                  
    tabsize=2,
}
\lstset{style=mystyle}

\renewcommand{\lstlistingname}{Código}

\ifCLASSINFOpdf

\else

\fi
\begin{document}
\onecolumn
\title{Proyecto final \\ Bases de Datos Avanzadas}
%
\author{Vicente Romero Andrade}

\markboth{Proyecto final Bases de Datos Avanzadas, Agosto~2021}%
{Shell \MakeLowercase{\textit{et al.}}: }
% The only time the second header will appear is for the odd numbered pages

\maketitle



\section{Objetivo}
% The very first letter is a 2 line initial drop letter followed

\IEEEPARstart{E}{l} objetivo de este proyecto es poner en práctica lo visto a lo largo 
del curso.

\section{Creación de la base de datos}
\subsection{Resumen de scripts creados}
\begin{table}[H]
  \centering
  \begin{longtable}{|c | c | c|} 
   \hline
   Num. Script & Nombre del script & Descripción \\ [0.5ex] 
   \hline
   1 & s-01-crea-directorios.sh & Crea los directorios que simularan los discos asdasdasdasdasda\\ 
   \hline
   2 & s-02-crea-pfile.sh & Crea el pfile y el archivo passwords \\
   \hline
   3 & s-03-crear-spfile.sql & Crea el archivo de spfile  \\
   \hline
   4 & s-04-crear-base.sql & Crea la base de datos asi como sus configuraciones basicas \\
   \hline
   5 & s-05-crear-diccionario-datos.sql & Crea el diccionario de datos de la base \\
   \hline
   6 & s-06-crear-tablespaces.sql & Crea los tablespaces de la base de datos \\
   \hline
  \end{longtable}
  \caption{Resumen scripts creados}
  \label{tabla:1}
\end{table}
\subsection{Configuraciones de la base de datos}
\begin{table}[H]
  \centering
  \begin{longtable}{|c | c | c|} 
   \hline
   Configuración & Descripción \\ [0.5ex] 
   \hline
   \multirow{4}{20em}{Número y ubicación de los archivos de control} & 3 archivo de control\\ 
   & /u01/app/oracle/oradata/VRAPROY/control01.ctl \\
   & /disk\_2/app/oracle/oradata/VRAPROY/control02.ctl \\
   & /disk\_3/app/oracle/oradata/VRAPROY/control03.ctl \\
   \hline
   \multirow{13}{20em}{Propuesta de grupos de REDO} & 3 grupos de redo\\
   & grupo 1\\
   & /u01/app/oracle/oradata/VRAPROY/redo01a.log\\
   & /disk\_2/app/oracle/oradata/VRAPROY/redo01b.log\\
   & /disk\_3/app/oracle/oradata/VRAPROY/redo01c.log\\
   & grupo 2\\
   & /u01/app/oracle/oradata/VRAPROY/redo02a.log\\
   & /disk\_2/app/oracle/oradata/VRAPROY/redo02b.log\\
   & /disk\_3/app/oracle/oradata/VRAPROY/redo02c.log\\
   & grupo 3\\
   & /u01/app/oracle/oradata/VRAPROY/redo03a.log\\
   & /disk\_2/app/oracle/oradata/VRAPROY/redo03b.log\\
   & /disk\_3/app/oracle/oradata/VRAPROY/redo03c.log\\
   \hline
   \multirow{2}{15em}{Propuesta de juego de caracteres} & AL32UTF8\\
   & Caracteres nacionales AL16UTF16\\ 
   \hline
   Tamaño del bloque de datos & default (La misma del SO)\\ 
   \hline
   \multirow{4}{15em}{Archivo passwords}
   & FILE=\$ORACLE\_HOME/dbs/orapwvraproy \\
   & FORMAT=12.2 \\
   & FORCE=Y \\
   & SYS=Hola1234! \\
    \hline
    \multirow{5}{28em}{Lista de parámetros que serán configurados al crear la base de datos} 
    & maxdatafiles 1024\\
    & maxloghistory 1\\
    & maxlogfiles 16\\
    & maxlogmembers 3\\ 
    & gestion local de extensiones \\
   \hline
  \end{longtable}
  \caption{Configuraciones de la base de datos}
  \label{tabla:2}
\end{table}
\subsection{Módulos del sistema}
\begin{table}[H]
  \centering
  \begin{longtable}{|c | l | c|} 
   \hline
   Nombre del módulo & Descripción & Usuario \\ [0.5ex] 
   \hline
   Administración de usuarios & Es el modulo que gestionara los datos de los usuarios asi como los pagos y planes de suscripción & admin\_usuario \\ 
   \hline
   Administración de multimedia & Es el modulo que gestionara los datos multimedia asi como el contenido relacionado a sus autores & admin\_multimedia \\  [1ex] 
   \hline
  \end{longtable}
  \caption{Módulos del sistema}
  \label{tabla:3}
\end{table}
\subsection{Diseño lógico de la base de datos}
\begin{table}[H]
  \centering
  \begin{longtable}{|c | c | c|} 
   \hline
   Nombre de la tabla & Nombre del módulo \\ [0.5ex] 
   \hline
   USUARIO & Administración de usuarios \\ 
   \hline
   CUENTA\_USUARIO & Administración de usuarios \\ 
   \hline
   TARJETA & Administración de usuarios \\ 
   \hline
   CARGO\_TARJETA & Administración de usuarios \\ 
   \hline
   PLAN\_SUSCRIPCION & Administración de usuarios \\ 
   \hline
   COSTO\_VIGENTE\_HISTORICO & Administración de usuarios \\ 
   \hline
   DISPOSITIVO & Administración de usuarios \\ 
   \hline
   PLAYLIST & Administración de usuarios \\ 
   \hline
   PLAYLIST\_CONTENIDO & Administración de multimedia \\ 
   \hline
   CONTENIDO\_MULTIMEDIA & Administración de multimedia \\ 
   \hline
   CONTENIDO\_MULTIMEDIA\_AUTOR & Administración de multimedia \\ 
   \hline
   AUTOR & Administración de multimedia \\ 
   \hline
   GENERO & Administración de multimedia \\ 
   \hline
   VIDEO & Administración de multimedia \\ 
   \hline
   AUDIO & Administración de multimedia \\ 
   \hline
   CONTENIDO\_SECCION & Administración de multimedia \\ 
   \hline
   REPRODUCCION & Administración de multimedia \\ 
   \hline
   COMENTARIO & Administración de multimedia \\ 
   [1ex] 
   \hline
  \end{longtable}
  \caption{Diseño lógico de la base de datos}
  \label{tabla:4}
\end{table}
\subsection{Esquema de indexado}
Se genera el esquema de indices
\begin{table}[H]
  \centering
  \begin{longtable}{| c | c | c | L{2.5cm} |} 
   \hline
   Nombre de la tabla & Nombre del índice & Tipo & Propósito \\ [0.0ex] 
   \hline
   USUARIO & USUARIO\_USERNAME\_IUK & UNIQUE & Asegura unicidad de username ademas de mejorar consulta usando lower \\
   \hline
   CUENTA\_USUARIO & CUENTA\_USUARIO\_CUENTA\_COMPARTIDA\_IUK & UNIQUE & Asegura unicidad de usuario propietario de cuenta y usuario compartido \\
   \hline
   DISPOSITIVO & DISPOSITIVO\_USUARIO\_ID\_IX & NON UNIQUE & Mejora desempeño de busqueda de dispositivos de un usuario \\
   \hline
   PLAYLIST & PLAYLIST\_NOMBRE\_USUARIO\_IUK & UNIQUE & Asegura que un usuario tenga playlist con nombres unicos \\
   \hline
   PLAYLIST\_USUARIO & PLAYLIST\_USUARIO\_USUARIO\_PLAYLIST\_IUK & UNIQUE & Asegura que un playlist fuese compartido una vez con un usuario \\
   \hline
   TARJETA & TARJETA\_USUARIO\_NUMERO\_IUK & UNIQUE & Asegura que un usuario solo registre una vez la tarjeta \\
   \hline
   CARGO\_TARJETA & CARGO\_TARJETA\_FOLIO\_IUK & UNIQUE & Asegura unicidad de folio de cargo y tarjeta \\
   \hline
   PLAN\_SUSCRIPCION & PLAN\_SUSCRIPCION\_DATOS\_IUK & UNIQUE & Asegura unicidad de datos de plan suscripción \\
   \hline
   COSTO\_VIGENTE\_HISTORICO & COSTO\_VIGENTE\_HISTORICO\_VIGENCIA\_IUK & UNIQUE & Asegura que solo exista un costo en cierta vigencia para un plan \\
   \hline
   PLAYLIST\_CONTENIDO & PLAYLIST\_CONTENIDO\_MULTIMEDIA\_PLAY\_IUK & UNIQUE & Asegura que solo existe una vez el contenido en el playlist \\
   \hline
   COMENTARIO & COMENTARIO\_USUARIO\_CONTENIDO\_IX & NON UNIQUE & Mejora desempeño de busqueda de comentarios de usuarios en un contenido \\
   \hline
   COMENTARIO & COMENTARIO\_USUARIO\_RESPUESTA\_IX & NON UNIQUE & Mejora desempeño de busqueda de respuestas a comentarios \\
   \hline
   CONTENIDO\_MULTIMEDIA & CONTENIDO\_MULTIMEDIA\_CLAVE\_IUK & UNIQUE & Asegura unicidad de la clave \\
   \hline
   CONTENIDO\_MULTIMEDIA & CONTENIDO\_MULTIMEDIA\_NOMBRE\_IX & NON UNIQUE & Mejora desempeño de busqueda por nombre \\
   \hline
   GENERO & GENERO\_NOMBRE\_IUK & UNIQUE & Mejora desempeño de busqueda por nombre y asegura unicidad \\
   \hline
   CONTENIDO\_MULTIMEDIA\_AUTOR & CONTENIDO\_MULTIMEDIA\_AUTOR\_AUTOR\_CONTENIDO\_IUK & UNIQUE & Asegura que el autor este asociado una vez al contenido multimedia \\
   \hline
   AUTOR & AUTOR\_NOMBRE\_REAL\_ARTISTICO\_IUK & UNIQUE & Asegura que el autor exista una vez con el nombre artistico \\
   \hline
   CONTENIDO\_SECCION & CONTENIDO\_SECCION\_SECUENCIA\_CONTENIDO\_IUK & UNIQUE & Asegura que una secuencia este asociado a solo un contenido \\
   \hline
   REPRODUCCION & REPRODUCCION\_DISPOSITIVO\_CONTENIDO\_IX & NON UNIQUE & Mejora busqueda de reproducciones de cada dispositivo \\ [1ex] 
   \hline
  \end{longtable}
  \caption{Esquema de indexado}
  \label{tabla:5}
\end{table}
\section{Diseño físico básico de la BD}
\subsection{Definición de tablespaces comunes a los módulos}
\begin{table}[H]
  \centering
  \begin{tabular}{|c|c|} 
   \hline
   Nombre del tablespace & Configuración \\ [0.5ex] 
   \hline
    \multirow{4}{4em}{system} 
    & tamaño inicial de 700M \\
    & reusar datafile \\
    & Autoextend de 1M ilimitado \\
    & ubicación datafile /u01/app/oracle/oradata/VRAPROY/system01.dbf \\ 
    \hline
    \multirow{4}{4em}{sysaux} 
    & tamaño inicial de 500M \\
    & reusar datafile \\
    & Autoextend de 1M ilimitado \\
    & ubicación datafile /u01/app/oracle/oradata/VRAPROY/sysaux01.dbf \\ 
    \hline
    \multirow{4}{4em}{users} 
    & tamaño inicial de 500M \\
    & Autoextend por default (1M) ilimitado \\
    & reusar datafile \\
    & ubicación datafile /disk\_2/app/oracle/oradata/VRAPROY/users.dbf \\ 
    \hline
    \multirow{5}{4em}{tempts1}
    & temporal \\ 
    & tamaño inicial de 20M \\
    & Autoextend de 640k ilimitado \\
    & reusar datafile \\
    & ubicación datafile /u01/app/oracle/oradata/VRAPROY/temp01.dbf \\
    \hline
    \multirow{4}{4em}{undotbs1}
    & undo \\ 
    & tamaño inicial de 200M \\
    & Autoextend de 512k ilimitado \\
    & reusar datafile \\
    & ubicación datafile /u01/app/oracle/oradata/VRAPROY/undotbs01.dbf \\
   \hline
    \multirow{5}{4em}{indexesTbs} & tamaño inicial de 200M \\
    & Autoextend de 10M hasta llegar al máximo de 300M \\
    & gestion local automatica de extensiones \\
    & gestion automatica de segmentos \\
    & ubicación datafile /disk\_3/app/oracle/oradata/VRAPROY/indexTbs01.dbf \\
   \hline
  \end{tabular}
  \caption{Configuración de tablespaces comunes a los módulos}
  \label{tabla:6}
\end{table}
\subsection{Definición de tablespaces por módulo}
\begin{table}[H]
  \centering
  \begin{tabular}{|c|c|c|} 
   \hline
   Nombre del tablespace & Objetivo / Beneficio & Tipo \\ [0.000ex] 
   \hline
   usersPayTbs&Guardar de manera segura datos de pagos&users\\
   \hline
   undoUsersPayTbs&Guardar datos undo de las transacciones de pagos&undo\\
   \hline
   usersTbs&Guardar datos de los usuarios&users\\
   \hline
  \end{tabular}
  \caption{Definición de tablespaces modulo administración de usuarios}
  \label{tabla:7}
\end{table}
\begin{table}[H]
  \centering
  \begin{tabular}{|c|c|c|} 
   \hline
   Nombre del tablespace & Objetivo / Beneficio & Tipo \\ [0.000ex] 
   \hline
   multimediaTbs&Guardar datos de los archivos multimedia&users\\
   \hline
   multimediaObjTbs&Guardar datos de los archivos clob/blob multimedia&users\\
   \hline
  \end{tabular}
  \caption{Definición de tablespaces modulo administración de multimedia}
  \label{tabla:8}
\end{table}
\subsection{Configuración de tablespaces por módulo}
\begin{table}[H]
  \centering
  \begin{tabular}{|c | c |} 
   \hline
   Nombre del tablespace & Configuración \\ [0.5ex] 
   \hline
   \multirow{6}{5.5em}{usersPayTbs} & tamaño inicial de 500M \\
    & Autoextend al llegar al máximo ilimitada \\
    & gestion local automatica de extensiones \\
    & gestion automatica de segmentos \\
    & Cifrado AES256 \\
    & ubicación datafile /disk\_3/app/oracle/oradata/VRAPROY/usersPay01.dbf \\
    \hline
   \multirow{6}{7em}{undoUsersPayTbs} & tamaño inicial de 50M \\
    & Autoextend al llegar al máximo ilimitada \\
    & gestion local automatica de extensiones \\
    & gestion automatica de segmentos \\
    & Cifrado AES256 \\
    & ubicación datafile /disk\_3/app/oracle/oradata/VRAPROY/usersPay01.dbf \\
    \hline
    \multirow{5}{4em}{usersTbs} & tamaño inicial de 500M \\
    & Autoextend al llegar al máximo ilimitada \\
    & gestion local automatica de extensiones \\
    & gestion automatica de segmentos \\
    & ubicación datafile /disk\_2/app/oracle/oradata/VRAPROY/usersTbs01.dbf \\
   \hline
  \end{tabular}
  \caption{Configuración tablespaces modulo administración de usuarios}
  \label{tabla:9}
\end{table}
\begin{table}[H]
  \centering
  \begin{tabular}{| c | c |} 
   \hline
   Nombre del tablespace & Configuración \\ [0.5ex] 
   \hline
   \multirow{7}{6em}{multimediaObjTbs} & tamaño inicial de 1G \\
    & Tamaño de bloque de 8k \\
    & Bigfile \\
    & Autoextend de 256M hasta llegar al máximo ilimitada \\
    & gestion local automatica de extensiones \\
    & gestion automatica de segmentos \\
    & ubicación datafile /disk\_4/app/oracle/oradata/VRAPROY/multimediaObjTbs.dbf \\
    \hline
    \multirow{5}{4em}{multimediaTbs} & tamaño inicial de 500M \\
    & Autoextend al llegar al máximo ilimitada \\
    & gestion local automatica de extensiones \\
    & gestion automatica de segmentos \\
    & ubicación datafile /disk\_3/app/oracle/oradata/VRAPROY/multimediaTbs01.dbf \\
   \hline
  \end{tabular}
  \caption{Configuración tablespaces modulo administración de multimedia}
  \label{tabla:10}
\end{table}
\subsection{Asignación de tablespace para tablas de cada módulo}
\begin{table}[H]
  \centering
  \begin{tabular}{|c|c|} 
   \hline
   Nombre de la tabla & Nombre del tablespace \\ [0.1ex] 
   \hline
   USUARIO & usersTbs\\
   \hline
   PLAN\_SUSCRIPCION & usersTbs\\
   \hline
   COSTO\_VIGENTE\_HISTORICO & usersTbs\\
   \hline
   CUENTA\_USUARIO & usersTbs\\
   \hline
   DISPOSITIVO & usersTbs\\
   \hline
   PLAYLIST & usersTbs\\
   \hline
   PLAYLIST\_CONTENIDO & usersTbs\\
   \hline
   TARJETA & usersPayTbs\\
   \hline
   CARGO\_TARJETA & usersPayTbs\\
   \hline
  \end{tabular}
  \caption{Asignación de tablespace para tablas del módulo de usuarios}
  \label{tabla:11}
\end{table}
\begin{table}[H]
  \centering
  \begin{tabular}{|c|c|} 
   \hline
   Nombre de la tabla & Nombre del tablespace \\ [0.1ex] 
   \hline
   COMENTARIO & multimediaTbs\\
   \hline
   REPRODUCCION & multimediaTbs\\
   \hline
   CONTENIDO\_MULTIMEDIA & multimediaTbs\\
   \hline
   CONTENIDO\_MULTIMEDIA\_AUTOR & multimediaTbs\\
   \hline
   AUTOR & multimediaTbs\\
   \hline
   CONTENIDO\_SECCION & multimediaTbs\\
   \hline
   GENERO & multimediaTbs\\
   \hline
   VIDEO & multimediaTbs\\
   \hline
   AUDIO & multimediaTbs\\
   \hline
  \end{tabular}
  \caption{Asignación de tablespace para tablas del módulo de multimedia}
  \label{tabla:12}
\end{table}
\subsection{Asignación de tablespace para índices de cada módulo}
\begin{table}[H]
  \centering
  \resizebox{\linewidth}{!}{\begin{tabular}{|c|c|c|c|c|} 
   \hline
   Nombre del índice & Tipo de índice & Nombre de la tabla & Nombre de la columna & Nombre del tablespace \\ [0.1ex] 
   \hline
   USUARIO\_USERNAME\_IUK & UNIQUE & USUARIO & USERNAME &  indexesTbs \\
   \hline
   CUENTA\_USUARIO\_CUENTA\_COMPARTIDA\_IUK & UNIQUE & CUENTA\_USUARIO & USUARIO\_PROPIETARIO\_ID, USUARIO\_ID &  indexesTbs \\
   \hline
   DISPOSITIVO\_USUARIO\_ID\_IX & NON UNIQUE & DISPOSITIVO & USUARIO\_ID &  indexesTbs \\
   \hline
   PLAYLIST\_NOMBRE\_USUARIO\_IUK & UNIQUE & PLAYLIST & USUARIO\_CREADOR\_ID &  indexesTbs \\
   \hline
   PLAYLIST\_USUARIO\_USUARIO\_PLAYLIST\_IUK & UNIQUE & PLAYLIST\_USUARIO & USUARIO\_ID, PLAYLIST\_ID &  indexesTbs \\
   \hline
   TARJETA\_USUARIO\_NUMERO\_IUK & UNIQUE & TARJETA & USUARIO\_ID, NUMERO &  indexesTbs \\
   \hline
   CARGO\_TARJETA\_FOLIO\_IUK & UNIQUE & CARGO\_TARJETA & TARJETA\_ID, FOLIO &  indexesTbs \\
   \hline
   PLAN\_SUSCRIPCION\_DATOS\_IUK & UNIQUE & PLAN\_SUSCRIPCION & CLAVE, NOMBRE &  indexesTbs \\
   \hline
   COSTO\_VIGENTE\_HISTORICO\_VIGENCIA\_IUK & UNIQUE & COSTO\_VIGENTE\_HISTORICO & PLAN\_SUSCRIPCION\_ID, FECHA\_INICIO, FECHA\_FIN &  indexesTbs \\
   \hline
   PLAYLIST\_CONTENIDO\_MULTIMEDIA\_PLAY\_IUK & UNIQUE & PLAYLIST\_CONTENIDO & PLAYLIST\_ID, CONTENIDO\_MULTIMEDIA\_ID &  indexesTbs \\
   
   \hline
  \end{tabular}}
  \caption{Asignación de tablespace para tablas del módulo de administración de usuarios}
  \label{tabla:13}
\end{table}
\begin{table}[H]
  \centering
  \resizebox{\linewidth}{!}{\begin{tabular}{|c|c|c|c|c|} 
   \hline
   Nombre del índice & Tipo de índice & Nombre de la tabla & Nombre de la columna & Nombre del tablespace \\ [0.1ex] 
   \hline
   COMENTARIO\_USUARIO\_CONTENIDO\_IX & NON UNIQUE & COMENTARIO & CONTENIDO\_MULTIMEDIA\_ID, USUARIO\_ID &  indexesTbs \\
   \hline
   COMENTARIO\_USUARIO\_RESPUESTA\_IX & NON UNIQUE & COMENTARIO & COMENTARIO\_RESPUESTA\_ID, USUARIO\_ID, COMENTARIO\_ID &  indexesTbs \\
   \hline
   CONTENIDO\_MULTIMEDIA\_CLAVE\_IUK & UNIQUE & CONTENIDO\_MULTIMEDIA & CLAVE &  indexesTbs \\
   \hline
   CONTENIDO\_MULTIMEDIA\_NOMBRE\_IUK & UNIQUE & CONTENIDO\_MULTIMEDIA & NOMBRE &  indexesTbs \\
   \hline
   GENERO\_NOMBRE\_IUK & UNIQUE & GENERO & NOMBRE &  indexesTbs \\
   \hline
   CONTENIDO\_MULTIMEDIA\_AUTOR\_AUTOR\_CONTENIDO\_IUK & UNIQUE & CONTENIDO\_MULTIMEDIA\_AUTOR & CONTENIDO\_MULTIMEDIA\_ID, AUTOR\_ID &  indexesTbs \\
   \hline
   AUTOR\_NOMBRE\_REAL\_ARTISTICO\_IUK & UNIQUE & AUTOR & NOMBRE, APELLIDOS, NOMBRE\_ARTISTICO &  indexesTbs \\
   \hline
   CONTENIDO\_SECCION\_SECUENCIA\_CONTENIDO\_IUK & UNIQUE & CONTENIDO\_SECCION & ID\_SECUENCIA, CONTENIDO\_MULTIMEDIA\_ID &  indexesTbs \\
   \hline
   REPRODUCCION\_DISPOSITIVO\_CONTENIDO\_IX & UNIQUE & REPRODUCCION & CONTENIDO\_MULTIMEDIA\_ID, DISPOSITIVO\_ID &  indexesTbs \\
   \hline
  \end{tabular}}
  \caption{Asignación de tablespace para tablas del módulo de multimedia}
  \label{tabla:14}
\end{table}
\subsection{Asignación de tablespaces para columnas CLOB/BLOB de cada módulo}
\begin{table}[H]
  \centering
  \resizebox{\linewidth}{!}{\begin{tabular}{|c|c|c|c|c|} 
   \hline
   Nombre de la columna CLOB/BLOB & Nombre de índice asociado a la columna CLOB/BLOB & Nombre de la tabla & Nombre del tablespace para la columna CLOB/BLOB & Nombre del tablespace para el índice de la columna CLOB/BLOB \\ [0.1ex] 
   \hline
   LETRA & N/A & AUDIO & multimediaObjTbs & N/A \\
   \hline
   CONTENIDO & N/A & CONTENIDO\_SECCION & multimediaObjTbs & N/A \\
   \hline
  \end{tabular}}
  \caption{Asignación de tablespaces para columnas CLOB/BLOB del módulo de multimedia}
  \label{tabla:15}
\end{table}
\section{Creación de usuarios}
\lstinputlisting[language=sql,caption=s-07-crear-usuarios.sql,label={lst:codigousuarios}]{../s-07-crear-usuarios.sql}
\section{Generación de código DDL}
Aqui no se añadira el codigo por cuestiones de tamaño
\section{Modos de conexión}
\lstinputlisting[language=sql,caption=s-09-crear-conexiones.sql,label={lst:codigoconexion}]{../s-09-crear-conexiones.sql}
\section{Habilitar la FRA}

\ifCLASSOPTIONcaptionsoff
  \newpage

\fi

\end{document}
