\documentclass[journal]{IEEEtran}
\usepackage[english]{babel}

\usepackage{amssymb, amsmath} %Paquetes matemáticos de la American Mathematical 
\usepackage[utf8]{inputenc}
\usepackage{graphicx}
\usepackage{float}
\usepackage{hyperref}
\usepackage{listings}
\usepackage{xcolor}

\definecolor{codegreen}{rgb}{0,0.6,0}
\definecolor{codegray}{rgb}{0.5,0.5,0.5}
\definecolor{codepurple}{rgb}{0.58,0,0.82}
\definecolor{backcolour}{rgb}{0.95,0.95,0.92}
% Definicio de estilo para el codigo fuente que se cita
\lstdefinestyle{mystyle}{
    backgroundcolor=\color{backcolour},   
    commentstyle=\color{codegreen},
    keywordstyle=\color{magenta},
    numberstyle=\tiny\color{codegray},
    stringstyle=\color{codepurple},
    basicstyle=\ttfamily\footnotesize,
    breakatwhitespace=false,         
    breaklines=true,                 
    captionpos=b,                    
    keepspaces=true,
    numbers=left,                    
    numbersep=5pt,                  
    showspaces=false,                
    showstringspaces=false,
    showtabs=false,                  
    tabsize=2,
}
\lstset{style=mystyle}

\renewcommand{\lstlistingname}{Código}

\ifCLASSINFOpdf

\else

\fi

\hyphenation{op-tical net-works semi-conduc-tor}


\begin{document}

\title{Ejercicio 1 - tema 4 \\ Administración de las estructuras de Memoria}
%
\author{Vicente Romero Andrade}

\markboth{Ejercicio 1 - tema 4 Administración de las estructuras de Memoria, Julio~2021}%
{Shell \MakeLowercase{\textit{et al.}}: }
% The only time the second header will appear is for the odd numbered pages

\maketitle


\IEEEpeerreviewmaketitle

\section{Objetivo}
% The very first letter is a 2 line initial drop letter followed

\IEEEPARstart{E}{l} objetivo es, Conocer y familiarizarse con las principales vistas y parámetros 
de la base de datos que muestran o contienen información relevante acerca del uso
de las diferentes áreas de la SGA.

\section{Desarrollo}
\subsection{C1.  Exploraciòn de las áreas de memoria}
\subsubsection{Crear script s-00-crea-usuario.sql que debe crear al usuario vra0401 y le otorgue los privilegios necesarios}

\begin{table}[H]
  \centering
  \begin{tabular}{||c |c| c||} 
   \hline
   Pregunta & Respuesta(inciso) &Explicación/justificación.\\ [1.0ex] 
   \hline
   P01 & B & Al ser un cierre normal se espera que ya no existan transacciones activas, en este caso se espera a que haga commit el usuario \\ \hline
   P02 & E & Aun esta la sesión activa por lo que se espera su cierre \\ \hline
   P03 & B & Se termina la sesión y se termina de ejecutar T5 \\ \hline
   P04 & A & Se ejecuta sin errores ya que la base de datos estaba en shutdown que es el estado final del cierre \\ \hline
   P05 & C & En esta fase de modo mount solo se puede acceder si el usuario es administrador \\ \hline
   P06 & C & Se ejecuta ya que estaba en modo mount que es una antes del modo open \\ \hline
   P07 & A & Es exitoso ya que esta montada y abierta la base de datos, es como su se hiciera un startup normal \\ \hline
   P08 & E & Como solo esta la sesión de un usuario y este no esta enmedio de una transacción en curso, se apaga completamente\\ \hline
   P09 & C & La base de datos se apago de forma correcta por lo que ya no se pueden hacer mas operaciones \\ \hline
   P10 & C & Se volvio a iniciar sesión con la base abierta por lo que se ejecutan las instrucciones correctamente \\ \hline
   P11 & D & Se hace rollback ya que asi se espera que actue la base \\ \hline
   P12 & A & Como se hizo rollback implicito, no se muestra ningun registro \\ \hline
   P13 & E & Se realiza de manera exitosa ya que no hay transacciones en espera \\  [1ex] 
   \hline
  \end{tabular}
  \caption{Tabla De respuestas}
  \label{tabla:1}
  \end{table}
\subsection{C2. Contenido del archivo eventos.log}
\begin{lstlisting}[language=bash, caption=eventos.log,label={lst:eventos_log}]
  sys@vrabda2> shutdown immediate
  Database closed.
  Database dismounted.
  ORACLE instance shut down.
  sys@vrabda2> startup
  ORACLE instance started.
  
  Total System Global Area  805303360 bytes                                       
  Fixed Size                  8901696 bytes                                       
  Variable Size             587202560 bytes                                       
  Database Buffers          201326592 bytes                                       
  Redo Buffers                7872512 bytes                                       
  Database mounted.
  Database opened.
  sys@vrabda2> shutdown transactional
  Database closed.
  Database dismounted.
  ORACLE instance shut down.
  sys@vrabda2> exit
  
  
  vra0301@vrabda2> select * from test2;
  
          ID                                                                      
  ----------                                                                      
  
  0 rows selected.
  \end{lstlisting}

\section{Conclusiones}
En este ejercicio se exploro el inicio y la detencion de la base de datos, fue util ya que
existen varias formas de llevar a cabo esta acción las cuales implican algunas acciones 
que pueden ocurrir segun sea el caso.
\ifCLASSOPTIONcaptionsoff
  \newpage

\fi

\end{document}
